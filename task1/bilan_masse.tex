\section{Bilan de masse du plant}

On cherche à calculer les quantités de \ce{CH4}, \ce{H2O} et d'air\footnotemark (respectivement $n_{i,\ce{CH4}}$, $n_{i,\ce{H2O}}$ et $n_{i,\text{air}}$, en moles) nécessaires pour produire $n_{f,\ce{NH3}}$\;\mole{} d'ammoniac, avec une température du réformeur primaire de $T$\;\kelvin{}.
\footnotetext{La composition de l'air étant : $78\%$ \ce{N2}, $21\%$ \ce{O2}, $1\%$ \ce{Ar}, en fraction molaire.}

Pour ce faire, nous décomposons le bilan en deux parties : tout d'abord, nous allons considérer les réactions se passant au sein du plant (réformeur primaire, réformeur secondaire, WGS et réacteur) et en déduire les quantités de matière nécessaires ; ensuite, nous ajouterons à ce premier bilan la masse de méthane brûlée dans le four pour contrebalancer l'effet endothermique des réactions du réformeur primaire.

\subsection{Bilan des réactions de synthèse}

Pour obtenir ce bilan de masse, nous devons résoudre les dépendances entre les entrées et sorties d'espèces, et les réactions qui se produisent entre ces espèces. Ces dépendances sont de deux types : linéaires --- les réactions chimiques --- et non linéaires --- les constantes d'équilibre thermodynamiques.

\subsubsection{Équations linéaires}

\begin{figure}[h]
    \begingroup
    \addtolength{\jot}{.5em}
    \begin{align}
      \cee{
         CH4 +  H2O &<=>  CO  + 3H2 \\
         CO  +  H2O &<=>  CO2 +  H2 \\
        2CH4 +  O2  &-> 2CO + 4H2 \\
         CO  +  H2O &->  CO2 +  H2\\
         N2  + 3H2  &-> 2NH3
      }
      \text.
    \end{align}
    \endgroup

    \caption{Les 5 équations de synthèse du plant.}
\end{figure}

Nous allons commencer par résoudre les relations linéaires. L'ensemble des entrées, sorties et réactions se décomposent de la manière suivante :
\begin{itemize}
  \item entrée de \ce{CH4} et \ce{H2O} ($n_{i,\ce{CH4}}$ et $n_{i,\ce{H2O}}$) ;
  \item réformeur primaire (réactions $R_1$ et $R_2$, incomplètes) ;
  \item entrée d'air ($n_{i,\text{air}}$) ;
  \item réformeur secondaire (réaction $R_3$, complète) ;
  \item water-gas shift (réaction $R_4$, complète) ;
  \item sortie de \ce{H2O} et \ce{CO2} ($n_{f,\ce{H2O}}$ et $n_{f,\ce{CO2}}$) ;
  \item synthèse de l'ammoniac (réaction $R_5$, complète) ;
  \item sortie de \ce{Ar} et \ce{NH3} ($n_{f,\ce{Ar}}$ et $n_{f,\ce{NH3}}$).
\end{itemize}
Au total, nous avons 12 variables : 3 entrées, 4 sorties et 5 réactions.

La conservation de la masse implique que la somme pondérée des entrées, sorties et réactions pour chaque réactif soit égale à zéro. Par exemple, pour \ce{H2O}, cela revient à :
\[
  n_{i,\ce{H2O}} - n_{f,\ce{H2O}} - R_1 - R_2 - R_4 = 0
  \text,
\]
tandis que pour le \ce{CH4}, nous avons :
\[
  n_{i,\ce{CH4}} - R_1 - 2R_3 = 0
  \text,
\]
etc.

Si l'on fait la même chose pour chacune des 9 espèces impliquées, on obtient un système\footnotemark de 9 équations linéaires homogènes à 12 inconnues. La résolution de ce système --- sous forme matricielle, via Matlab --- nous donne un espace vectoriel $V$ de dimension 3, qui prend en compte toutes les dépendances linéaires entre les entrées, sorties et réactions.
\footnotetext{Le système en question est représenté sous forme matricielle dans l'annexe \ref{appendix:matrix}.}

Pour obtenir une solution unique, il faut donc fournir trois équations supplémentaires pour réduire cet espace à un point. Une de ces équations est linéaire : il s'agit simplement d'égaler $n_{f,\ce{NH3}}$ au nombre de moles de \ce{NH3} que l'on veut produire.

\subsubsection{Équations non-linéaires}

Les deux dernières équations, comme nous l'avons dit, sont non-linéaires et correspondent aux équilibres thermodynamiques des espèces en présence dans le réformeur primaire (réactions $R_1$ et $R_2$).

À chacune des deux réactions chimiques concernées correspond un quotient réactionnel et une constante d'équilibre, ici $K_1$ et $K_2$. Le calcul détaillé de la valeur ces constantes est expliqué dans annexe \ref{appendix:const_eq}.

À l'aide d'un tableau d'avancement des réactions, nous allons déterminer la quantité de réactifs et de produits au début et à l'équilibre des réactions 1 et 2 :
\begin{center}
  \begin{tabular}{lccccc}
    &  \ce{CH4} & \ce{H2O} & \ce{CO} & \ce{H2} & \ce{CO2}  \\
    \hline
    $n_\text{init}$
    & $n_{i,\ce{CH4}}$ & $n_{i,\ce{H2O}}$ & $R_{1}$ & 0 & 0  \\
    $n_\text{eq}$
    & $n_{i,\ce{CH4}}-R_1$ & $n_{i,\ce{H2O}}-R_1-R_2$ & $R_1-R_2$ & $3R_1+R_2$ & $R_2$
  \end{tabular}
\end{center}

On voit ici que $R_1$ et $R_2$ symbolisent le degré d'avancement des réactions du réformeur primaire.
%
Les valeurs obtenues dans les deux tableaux d'avancement nous permettent de calculer l'expression des activités des 5 différentes espèces en présence. Soit le gaz $\ce{X}$ et $p_\ce{X}$ sa pression partielle ; son activité est la suivante :
\[
  a_\ce{X} = \frac{p_\ce{X}}{p_\text{tot}} = \frac{n_\ce{X}}{n_\text{tot}} \frac{p_\text{tot}}{p^0}
  \text,
\]
avec $p_\text{tot}$ la pression moyenne du réacteur qui vaut \unit{29}{bar}\footnotemark, $n_\ce{X}$ le nombre de moles de \ce{X} en présence, $n_\text{tot}$ le nombre de moles total en présence --- ici $(n_{i,\ce{CH4}} + n_{i,\ce{H2O}} + 2R_1)$ --- et $p^0$ la pression de référence qui vaut \unit{1}{bar}.
\footnotetext{Information trouvée sur le forum du projet.}

À partir de cela, on obtient les équations suivantes :
\begin{align*}
K_1=\frac{a_\ce{CO}(a_\ce{H2})^3}{a_\ce{CH4}a_\ce{H2O}}&=\frac{(R_1-R_2)(3R_1+R_2)^3}{(n_{i,\ce{CH4}}-R_1)(n_{i,\ce{H2O}}-R_1-R_2)}
\left(\frac{p_\text{tot}}{(n_{i,\ce{CH4}} + n_{i,\ce{H2O}} + 2R_1) p^0}\right)^2 \\
K_2=\frac{a_\ce{H2}a_\ce{CO2}}{a_\ce{CO}a_\ce{H2O}}&=\frac{(3R_1+R_2)(R_2)}{(R_1-R_2)(n_{i,\ce{H2O}}-R_1-R_2)}
\end{align*}

Enfin, l'ajout de ces deux dernières équations à notre système et leur résolution\footnotemark{} réduit à zéro la dimension de l'espace $V$, fournissant la solution au problème, c'est-à-dire la valeur des différents fluxs d'entrée et de sortie, ainsi que les coefficients des réactions.
\footnotetext{Si les équations linéaires étaient encore relativement facile à résoudre à la main, les équations que nous avons ici sont d'un degré bien trop élevé et nécessitent un outil numérique --- dans notre cas Matlab.}

\subsection{Bilan de la combustion du méthane}

\[
    \Delta H_\text{RP} = R_1 \Delta H_{m,\text{R1}} + R_2 \Delta H_{m,\text{R2}}
    \text.
\]

Lorsque l'on calcule la variation d'enthalpie $\Delta H_\text{RP}$ des deux réactions se produisant au sein du réformeur primaire, pondérée par les nombres de moles qui réagissent, on se rend compte que celle-ci est positive : le bilan énergétique du réformeur primaire est endothermique. Il en découle que, pour y maintenir une température $T$ constante, il est nécessaire d'injecter une quantité de chaleur $Q = \Delta H_\text{RP}$; c'est le rôle du four. L'inconnue que nous cherchons est $n_\text{four}$, le nombre de moles de méthane à brûler.

Pour commencer, nous allons considérer en détail $\Delta H_\text{RP}$, la variation d'enthalpie dans le réformeur primaire. Celle-ci se décompose en deux, une partie provenant de la réaction 1, l'autre provenant de la réaction 2 :
\[
    \Delta H_\text{RP} = R_1 \Delta H_{m,\text{R1}} + R_2 \Delta H_{m,\text{R2}}
    \text.
\]

Passons au four : dans celui-ci, nous brûlons du méthane selon la réaction de combustion
\[\ce{CH4 + O2 -> CO2 + 2H2O}\text,\]
avec $\Delta H_\text{comb} < 0$ et un rendement thermique $\eta_\text{th} = 75\%$. La chaleur fournie par la combustion de $n$ moles de méthane dans le four est donc :
\[
    Q = - n_\text{four} \Delta H_{m,\text{comb}} \cdot \eta_\text{th}
    \text.
\]

La relation $Q = \Delta H_\text{RP}$ nous permet de résoudre pour $n_\text{four}$ :
\begin{align*}
    - n_\text{four} \Delta H_{m,\text{comb}} \cdot \eta_\text{th} &= R_1 \Delta H_{m,\text{R1}} + R_2 \Delta H_{m,\text{R2}} \\
    n_\text{four} &= -\frac{R_1 \Delta H_{m,\text{R1}} + R_2 \Delta H_{m,\text{R2}}}{\Delta H_{m,\text{comb}} \cdot \eta_\text{th}}
    \text.
\end{align*}
Notons qu'il s'agit à la fois de la quantité de méthane consommée et --- pensons à la planète --- de \ce{CO2} produite.

\subsection{Bilan total}

Le bilan total se calcule simplement en additionnant les valeurs trouvées dans les deux bilans précédents :
\begin{center}
    \renewcommand{\arraystretch}{1.3}
    \begin{tabular}{ccc|c}
        \ce{CH4} & \ce{H2O} & air & \ce{CO2} (rejeté) \\
        \hline
        $(n_{i,\ce{CH4}} + n_\text{four}) M_\ce{CH4}$ & $n_{i,\ce{H2O}} M_\ce{H2O}$ & $n_{i,\text{air}} M_\text{air}$ & $(n_{f,\ce{CO2}} + n_\text{four}) M_\ce{CO2}$ \\
        \hline
        \color{gray}$\unit{626.00}{\ton} + \unit{75.45}{\ton}$ & \color{gray}\unit{275.2}{\ton} & \color{gray}\unit{1639}{\ton} & \color{gray}$\unit{1717.58}{\ton} + \unit{207.02}{\ton}$ \\
        $\unit{701.45}{\ton}$ & \unit{275.2}{\ton} & \unit{1639}{\ton} & $ \unit{1924.61}{\ton}$

    \end{tabular}
\end{center}
Ces valeurs sont données pour \unit{1500}{\ton} d'ammoniac et une température de \unit{1000}{\kelvin}.
