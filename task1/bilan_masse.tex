\section{Bilan de masse du plant}

On cherche à calculer les quantités de \ce{CH4}, \ce{H2O} et d'air\footnotemark (respectivement $n_i(\ce{CH4})$, $n_i(\ce{H2O})$ et $n_i(\text{air})$, en moles) nécessaires pour produire $n_f(\ce{NH3})$\;\mole{} d'ammoniac, avec une température du réformeur primaire de $T$\;\kelvin{}.
\footnotetext{La composition de l'air étant : $78\%$ \ce{N2}, $21\%$ \ce{O2}, $1\%$ \ce{Ar}, en fraction molaire.}

Pour ce faire, nous décomposons le bilan en deux parties : tout d'abord, nous allons considérer les réactions se passant au sein du plant (réformeur primaire, réformeur secondaire, WGS et réacteur) et en déduire les quantités de matière nécessaires ; ensuite, nous ajouterons à ce premier bilan la masse de méthane utilisée pour chauffer les réactifs à la température $T$ du réformeur primaire.


\subsection{Bilan des réactions de synthèse}

Pour obtenir ce bilan de masse, nous devons résoudre les dépendances entre les entrées et sorties d'espèces, et les réactions qui se produisent entre ces espèces. Ces dépendances sont de deux types : linéaires --- les réactions chimiques --- et non linéaires --- les constantes d'équilibre thermodynamiques.

Nous allons commencer par résoudre les relations linéaires. L'ensemble des entrées, sorties et réactions se décomposent de la manière suivante :
\begin{itemize}
  \item entrée de \ce{CH4} et \ce{H2O} ($n_i(\ce{CH4})$ et $n_i(\ce{H2O})$) ;
  \item réformeur primaire (réactions $R_1$ et $R_2$, incomplètes) ;
  \item entrée d'air ($n_i(\text{air})$) ;
  \item réformeur secondaire (réaction $R_3$, complète) ;
  \item water-gas shift (réaction $R_4$, complète) ;
  \item sortie de \ce{H2O} et \ce{CO2} ($n_f(\ce{H2O})$ et $n_f(\ce{CO2})$) ;
  \item synthèse de l'ammoniac (réaction $R_5$, complète) ;
  \item sortie de \ce{Ar} et \ce{NH3} ($n_f(\ce{Ar})$ et $n_f(\ce{NH3})$).
\end{itemize}
Au total, nous avons 12 variables : 3 entrées, 4 sorties et 5 réactions.

La conservation de la masse implique que la somme des entrées, sorties et réactions pour chaque réactif soit égale à zéro. Par exemple, pour \ce{H2O}, cela revient à :
\[
  n_i(\ce{H2O}) - n_f(\ce{H2O}) - R_1 - R_2 - R_4 = 0
\]
Si l'on fait la même chose pour chacune des 9 espèces impliquées, on obtient un système de 9 équations linéaires homogènes à 12 inconnues. La résolution de ce système nous donne un espace vectoriel $V$ de dimension 3, qui prend en compte toutes les dépendances linéaires entre les entrées, sorties et réactions.

Ces 12 valeurs n'étant pas totalement indépendantes, faire le bilan de masse revient à résoudre les dépendances (linéaires et non linéaires) entre 

On peut considérer chacune de ces 12 grandeurs comme étant les coefficients de vecteurs dans un espace $V \in \mathbb{R}^9$ représentant des flux des 9 espèces chimiques différentes qui apparaissent dans le plant. Ainsi, un vecteur $(1, 0, \dots, 0)^T$ pourrait correspondre à une entrée de \ce{CH4}, et un autre vecteur $(0, \dots, -1)^T$ à une sortie de \ce{NH3}. Une réaction serait alors également représentée sous la forme d'un vecteur (par exemple, $R_1$ : $(-1, -1, 3, 1, 0, \dots, 0)^T$).

De cette manière, on peut manipuler algébriquement les 12 \og{}flux\fg{} et résoudre les dépendances linéaires entre ceux-ci. Pour cela, il suffit de considérer une matrice $9\times12$ représentant l'espace $V$

Dans un premier temps, si l'on omet de considérer que les réactions se produisant dans le réformeur primaire ne sont pas complètes, 

Pour obtenir la solution à ce problème, il nous faut, dans un premier temps, résoudre les relations linéaires entre les différentes inconnues. Mathématiquement, cela correspond à obtenir une base de l'espace vectoriel des solutions.

Dans l'ordre, les colonnes sont : $n_i(\ce{CH4})$, $n_i(\ce{H2O})$, $n_i(\text{air})$, $n_f(\ce{H2O})$, $n_f(\ce{CO2})$, $n_f(\ce{Ar})$, $n_f(\ce{NH3})$, $R_1$, $R_2$, $R_3$, $R_4$ et $R_5$.
\[
\left(
\begin{array}{*{12}c}
  1 & 0 & 0 & 0 & 0 & 0 & 0 & -1 & 0 & -2 & 0 & 0 \\
  0 & 1 & 0 & -1 & 0 & 0 & 0 & -1 & -1 & 0 & -1 & 0 \\
  0 & 0 & 0 & 0 & 0 & 0 & 0 & 3 & 1 & 4 & 1 & -3 \\
  0 & 0 & 0 & 0 & 0 & 0 & 0 & 1 & -1 & 2 & -1 & 0 \\
  0 & 0 & 0 & 0 & -1 & 0 & 0 & 0 & 1 & 0 & 1 & 0 \\
  0 & 0 & .21 & 0 & 0 & 0 & 0 & 0 & 0 & -1 & 0 & 0 \\
  0 & 0 & .78 & 0 & 0 & 0 & 0 & 0 & 0 & 0 & 0 & -1 \\
  0 & 0 & .01 & 0 & 0 & -1 & 0 & 0 & 0 & 0 & 0 & 0 \\
  0 & 0 & 0 & 0 & 0 & 0 & -1 & 0 & 0 & 0 & 0 & 2
\end{array}
\right)
\]


\subsection{Bilan de la combustion du méthane}

