\section{Bilan de masse du plant}

On cherche à calculer les quantités de \ce{CH4}, \ce{H2O} et d'air\footnotemark (respectivement $n_i(\ce{CH4})$, $n_i(\ce{H2O})$ et $n_i(\text{air})$, en moles) nécessaires pour produire $n_f(\ce{NH3})$\;\mole{} d'ammoniac, avec une température du réformeur primaire de $T$\;\kelvin{}.
\footnotetext{La composition de l'air étant : $78\%$ \ce{N2}, $21\%$ \ce{O2}, $1\%$ \ce{Ar}, en fraction molaire.}

Pour ce faire, nous décomposons le bilan en deux parties : tout d'abord, nous allons considérer les réactions se passant au sein du plant (réformeur primaire, réformeur secondaire, WGS et réacteur) et en déduire les quantités de matière nécessaires ; ensuite, nous ajouterons à ce premier bilan la masse de méthane utilisée pour chauffer les réactifs à la température $T$ du réformeur primaire.


\subsection{Bilan des réactions de synthèse}

Pour obtenir ce bilan de masse, nous devons résoudre les dépendances entre les entrées et sorties d'espèces, et les réactions qui se produisent entre ces espèces. Ces dépendances sont de deux types : linéaires --- les réactions chimiques --- et non linéaires --- les constantes d'équilibre thermodynamiques.

\subsubsection{Équations linéaires}

Nous allons commencer par résoudre les relations linéaires. L'ensemble des entrées, sorties et réactions se décomposent de la manière suivante :
\begin{itemize}
  \item entrée de \ce{CH4} et \ce{H2O} ($n_i(\ce{CH4})$ et $n_i(\ce{H2O})$) ;
  \item réformeur primaire (réactions $R_1$ et $R_2$, incomplètes) ;
  \item entrée d'air ($n_i(\text{air})$) ;
  \item réformeur secondaire (réaction $R_3$, complète) ;
  \item water-gas shift (réaction $R_4$, complète) ;
  \item sortie de \ce{H2O} et \ce{CO2} ($n_f(\ce{H2O})$ et $n_f(\ce{CO2})$) ;
  \item synthèse de l'ammoniac (réaction $R_5$, complète) ;
  \item sortie de \ce{Ar} et \ce{NH3} ($n_f(\ce{Ar})$ et $n_f(\ce{NH3})$).
\end{itemize}
Au total, nous avons 12 variables : 3 entrées, 4 sorties et 5 réactions.

La conservation de la masse implique que la somme des entrées, sorties et réactions pour chaque réactif soit égale à zéro. Par exemple, pour \ce{H2O}, cela revient à :
\[
  n_i(\ce{H2O}) - n_f(\ce{H2O}) - R_1 - R_2 - R_4 = 0
\]
Si l'on fait la même chose pour chacune des 9 espèces impliquées, on obtient un système de 9 équations linéaires homogènes à 12 inconnues. La résolution de ce système --- sous forme matricielle, via matlab --- nous donne un espace vectoriel $V$ de dimension 3, qui prend en compte toutes les dépendances linéaires entre les entrées, sorties et réactions.

Pour obtenir une solution unique, il faut donc fournir trois équations supplémentaires pour réduire cet espace à un point. Une de ces équations est linéaire : il s'agit simplement d'égaler $n_f(\ce{NH3})$ au nombre de moles de \ce{NH3} que l'on veut produire.

\subsubsection{Équations non-linéaires}

Les deux dernières équations, comme nous l'avons dit, sont non-linéaires et correspondent aux équilibres thermodynamiques des espèces en présence dans le réformeur primaire (réactions $R_1$ et $R_2$).

À chacune des deux réactions chimiques concernées correspond un quotient réactionnel et une constante d'équilibre, ici $K_1$ et $K_2$. Le calcul détaillé de la valeur ces constantes est \TODO{} fourni en annexe.

À l'aide d'un tableau d'avancement des réactions, nous allons déterminer la quantité de réactifs et de produits au début et à l'équilibre des réactions 1 et 2 :
\begin{center}
  \begin{tabular}{lcccc}
    &  \ce{CH4} & \ce{H2O} & \ce{CO} & \ce{H2}  \\
    \hline
    $n_\text{init}$
    & $n_i(\ce{CH4})$ & $n_i(\ce{H2O})$ & 0 & 0  \\
    $n_\text{eq}$
    & $n_i(\ce{CH4})-R_1$ & $n_i(\ce{H2O})-R_1$ & $R_1$ & $3R_1$
  \end{tabular}
\end{center}
\begin{center}
  \begin{tabular}{lcccc}
    &  \ce{CO} & \ce{H2O} & \ce{CO2} & \ce{H2}  \\
    \hline
    $n_\text{init}$
    & $R_1$ & $n_i(\ce{H2O}) - R_1$ & 0 & 0  \\
    $n_\text{eq}$
    & $R_1-R_2$ & $n_i(\ce{H2O})-R_1-R_2$ & $R_2$ & $R_2$
  \end{tabular}
\end{center}
On voit ici que $R_1$ et $R_2$ symbolisent le degré d'avancement des réactions du réformeur primaire.
%
Les valeurs obtenues dans les deux tableaux d'avancement nous permettent de calculer l'expression des activités des 5 différentes espèces en présence. Soit le gaz $\ce{X}$ et $p_\ce{X}$ sa pression partielle ; son activité est la suivante :
\[
  a_\ce{X} = \frac{p_\ce{X}}{p_\text{tot}} = \frac{n_\ce{X}}{n_\text{tot}} \frac{p_\text{tot}}{p^0}
  \text,
\]
avec $p_\text{tot}$ la pression moyenne du réacteur qui vaut \unit{29}{bar}\footnotemark, $n_\ce{X}$ le nombre de moles de \ce{X} en présence, $n_\text{tot}$ le nombre de moles total en présence --- ici $n_i(\ce{CH4}) + n_i(\ce{H2O}) + 2R_1$ --- et $p^0$ la pression de référence qui vaut \unit{1}{bar}.
\footnotetext{Information trouvée sur le forum du projet.}

À partir de cela, on obtient les équations suivantes :
\begin{align*}
K_1=\frac{a_\ce{CO}(a_\ce{H2})^3}{a_\ce{CH4}a_\ce{H2O}}&=\frac{(R_1-R_2)(3R_1+R_2)^3}{(n_i(\ce{CH4})-R_1)(n_i(\ce{H2O})-R_1-R_2)}
\left(\frac{p_\text{tot}}{(n_i(\ce{CH4}) + n_i(\ce{H2O}) + 2R_1) p^0}\right)^2 \\
K_2=\frac{a_\ce{H2}a_\ce{CO2}}{a_\ce{CO}a_\ce{H2O}}&=\frac{(3R_1+R_2)(R_2)}{(R_1-R_2)(n_i(\ce{H2O})-R_1-R_2)}
\end{align*}

Enfin, l'ajout de ces deux dernières équations à notre système et leur résolution\footnotemark{} réduit à zéro la dimension de l'espace $V$, fournissant la solution au problème, c'est-à-dire la valeur des différents fluxs d'entrée et de sortie, ainsi que les coefficients des réactions.
\footnotetext{Si les équations linéaires étaient encore relativement facile à résoudre à la main, les équations que nous avons ici sont d'un degré bien trop élevé et nécessitent un outil numérique --- dans notre cas Matlab.}



% Dans l'ordre, les colonnes sont : $n_i(\ce{CH4})$, $n_i(\ce{H2O})$, $n_i(\text{air})$, $n_f(\ce{H2O})$, $n_f(\ce{CO2})$, $n_f(\ce{Ar})$, $n_f(\ce{NH3})$, $R_1$, $R_2$, $R_3$, $R_4$ et $R_5$.
% \[
% \left(
% \begin{array}{*{12}c}
%   1 & 0 & 0 & 0 & 0 & 0 & 0 & -1 & 0 & -2 & 0 & 0 \\
%   0 & 1 & 0 & -1 & 0 & 0 & 0 & -1 & -1 & 0 & -1 & 0 \\
%   0 & 0 & 0 & 0 & 0 & 0 & 0 & 3 & 1 & 4 & 1 & -3 \\
%   0 & 0 & 0 & 0 & 0 & 0 & 0 & 1 & -1 & 2 & -1 & 0 \\
%   0 & 0 & 0 & 0 & -1 & 0 & 0 & 0 & 1 & 0 & 1 & 0 \\
%   0 & 0 & .21 & 0 & 0 & 0 & 0 & 0 & 0 & -1 & 0 & 0 \\
%   0 & 0 & .78 & 0 & 0 & 0 & 0 & 0 & 0 & 0 & 0 & -1 \\
%   0 & 0 & .01 & 0 & 0 & -1 & 0 & 0 & 0 & 0 & 0 & 0 \\
%   0 & 0 & 0 & 0 & 0 & 0 & -1 & 0 & 0 & 0 & 0 & 2
% \end{array}
% \right)
% \]


\subsection{Bilan de la combustion du méthane}



\subsection{Bilan total}
