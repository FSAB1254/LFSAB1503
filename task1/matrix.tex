\section{Système linéaire du bilan de masse}\label{appendix:matrix}

Pour éventuellement aider à comprendre le fonctionnement du bilan de masse, nous fournissons ici le système, sous forme matriciel, qui est à résoudre pour obtenir l'espace vectoriel $V$.

Dans l'ordre, les lignes de la matrice correspondent aux composés suivants : \ce{CH4}, \ce{H2O}, \ce{O2}, \ce{N2}, \ce{Ar}, \ce{CO},  \ce{CO2}, \ce{H2} et \ce{NH3}.
\[
\left(
\begin{array}{*{12}c}
  1 & 0 & 0 & 0 & 0 & 0 & 0 & -1 & 0 & -2 & 0 & 0 \\
  0 & 1 & 0 & -1 & 0 & 0 & 0 & -1 & -1 & 0 & -1 & 0 \\
  0 & 0 & 0 & 0 & 0 & 0 & 0 & 3 & 1 & 4 & 1 & -3 \\
  0 & 0 & 0 & 0 & 0 & 0 & 0 & 1 & -1 & 2 & -1 & 0 \\
  0 & 0 & 0 & 0 & -1 & 0 & 0 & 0 & 1 & 0 & 1 & 0 \\
  0 & 0 & .21 & 0 & 0 & 0 & 0 & 0 & 0 & -1 & 0 & 0 \\
  0 & 0 & .78 & 0 & 0 & 0 & 0 & 0 & 0 & 0 & 0 & -1 \\
  0 & 0 & .01 & 0 & 0 & -1 & 0 & 0 & 0 & 0 & 0 & 0 \\
  0 & 0 & 0 & 0 & 0 & 0 & -1 & 0 & 0 & 0 & 0 & 2
\end{array}
\right)
\left(
\begin{array}{*{1}c}
  n_{i,\ce{CH4}} \\ n_{i,\ce{H2O}} \\ n_{i,\text{air}} \\ n_{f,\ce{H2O}} \\ n_{f,\ce{CO2}} \\ n_{f,\ce{Ar}} \\ n_{f,\ce{NH3}} \\ R_1 \\ R_2 \\ R_3 \\ R_4 \\ R_5
\end{array}
\right)
= 0
\]
