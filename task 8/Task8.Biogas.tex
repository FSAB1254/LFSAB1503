\documentclass[10pt,a4paper]{article}
\usepackage[utf8]{inputenc}
\usepackage[francais]{babel}
\usepackage[T1]{fontenc}
\usepackage{amsmath}
\usepackage{amsfonts}
\usepackage{amssymb}
\usepackage{graphicx}
\usepackage[squaren,Gray]{SIunits} % Physical units rendering
\usepackage{sistyle}
\usepackage[autolanguage]{numprint}
%\usepackage{xfrac}
\usepackage{bm}
\usepackage{color} % Colors in text
\usepackage[version=3]{mhchem} % Chemical reactions
\title{Tâche 8 : Biogaz}
\author{Sliti Abbas, Paulus Léa, Lambein Xavier}
\date{06 Decembre 2014}



\begin{document}

\maketitle

Le biogaz est une alternative au méthane qui est déjà utiliser et qui ne cesse de progresser. C'est un gaz produit par fermentation de matières organiques animales ou végétales en l'absence d'oxygène dans le but de remplacer l'utilisation du méthane pur. Ce procédé se fait naturellement et spontanément ou artificiellement par des usines spécialisées  Le biogaz est consititué de méthane (50\% à 70\%), de \ce{CO2}, de \ce{H2S} et d'impuretés. 

L'avantage majeur de ce gaz est qu'il n'augmente pas la quantité de \ce{CO2} dans l'atmosphère. En effet, le rejet de \ce{CO2} provenant de l'utilisation du biogaz équivaut à celui consommé par les végétaux dont ce biogaz est issu. Son impact environnemental est donc considérable.

Cependant, le seul composé qui permet de produire de l'énergie est le méthane. Il faut donc le séparer des autres constituants, ce qui implique des méthodes de séparation coûteuses en énergie ou en support matériel. Rappelons néanmoins que l'extraction du méthane demande également de l'énergie. Il faut comparer les méthodes de production de biogaz et de méthane.

Autre avantage du biogaz, si celui-ci provient directement d'une fermentation spontanée et naturelle (décharges à déchets organiques, marais, ...), ceci contribue à diminuer la quantité de \ce{CH4} dans l'atmosphère : le méthane est beaucoup plus favorable au réchauffement climatique que le \ce{CO2} (jusqu'à 23 fois plus).

Aussi, si le méthane est importé par voies extérieur, il faut le transporter vers l'usine où il est produit alors que le biogaz peut être produit localement avec de faibles coûts de transport (voire même aucun s'il est produit par l'usine elle-même!).

L'inconvénient majeur du biogaz est sa demande importante de déchets organiques ...





\end{document}