\documentclass[10pt,a4paper]{article}
\usepackage[utf8]{inputenc}
\usepackage[francais]{babel}
\usepackage[T1]{fontenc}
\usepackage{amsmath}
\usepackage{amsfonts}
\usepackage{amssymb}
\usepackage{graphicx}
\usepackage[squaren,Gray]{SIunits} % Physical units rendering
\usepackage{sistyle}
\usepackage[autolanguage]{numprint}
%\usepackage{xfrac}
\usepackage{bm}
\usepackage{color} % Colors in text
\usepackage[version=3]{mhchem} % Chemical reactions
\title{Tâche 8 : Electrolyse}
\author{Groupe 1254}

\begin{document}

\maketitle

Une alternative au méthane envisageable et qui fait l'objet de nombreuses recherches actuellement est l’électrolyse de l'eau. L’électrolyse décompose la molécule d'eau en oxygène et en hydrogène par le passage d'un courant électrique dans l'eau. Alors pourquoi produire de l'hydrogène par électrolyse ? 

Le principal avantage de cette méthode est qu'elle n'émet pas de $\ce CO2$. Cependant ce processus demandant un apport énergétique en électricité, il faut tout de même prendre en compte comment l'électricité utilisée est produite. Durant notre projet, nous avons été amenés à réaliser un labo sur l’électrolyse de l'eau et d'étudier ce système : le rendement était \unit{82.3}{\%}, ce qui reste un bon rendement et enfin en calculant la puissance, nous avons pu déduire la puissance nécessaire pour produire \unit{1500}{t/jour} d'ammoniac. Celle-ci est de \unit{5.7}{GW} ce qui équivaut à la production de 5 centrales nucléaires ou \unit{2850}{Ha} de panneaux photovoltaïques sans prendre en compte le pompage de l'eau. Il est donc inconcevable de remplacer le méthane pour produire de l'hydrogène par électrolyse de l'eau. Mais, nous pouvons contribuer à diminuer la consommation de méthane et donc d'émettre moins de $\ce CO2$ en utilisant de l’électricité produit de manière propre (sans émettre de $\ce CO2$) par des énergies renouvelables telles que l'énergie éolien, solaire (à l'aide de panneaux photovoltaïques),...  .

Un autre aspect qui nous pousserait à envisager ce mode de production est la production locale où les coûts de transports en seraient donc diminués. 

En termes de matières premières, le méthane est une source non-renouvelable qui s'épuisera d'ici une soixantaine d'années alors que l’électrolyse ne présente, pour l'instant, pas de problèmes d'approvisionnement.
\end{document}