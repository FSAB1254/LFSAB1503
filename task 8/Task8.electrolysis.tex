\documentclass[10pt,a4paper]{article}
\usepackage[utf8]{inputenc}
\usepackage[francais]{babel}
\usepackage[T1]{fontenc}
\usepackage{amsmath}
\usepackage{amsfonts}
\usepackage{amssymb}
\usepackage{graphicx}
\usepackage[squaren,Gray]{SIunits} % Physical units rendering
\usepackage{sistyle}
\usepackage[autolanguage]{numprint}
%\usepackage{xfrac}
\usepackage{bm}
\usepackage{color} % Colors in text
\usepackage[version=3]{mhchem} % Chemical reactions
\title{Tâche 8 : Electrolyse}
\author{Sliti Abbas}

\begin{document}

\maketitle

Une alternative au méthane envisageable et qui fait l'objet de nombreuses recherches actuellement est l'electrolyse de l'eau. L'electrolyse décompose la molécule d'eau en oxygène et en hydrogène par le passage d'un courant électrique dans l'eau. Alors pourquoi produire de l'hydrogène par electrolyse ? 

Le principal avantage de cette méthode est qu'elle n'émette pas de $\ce CO2$ mais tout dépend de comment l'électricité utilisé a été produite, c'est vraiment le paramètre le plus important à prendre en compte. Durant notre projet, nous avons été amenés à réaliser un labo sur l'electrolyse de l'eau et d'étudier ce sytème : le rendement était 82.3 \%, ce qui reste un bon rendement et enfin en calculant la puissance, nous pouvions en déduire la puissance nécessaire pour produire 1500t/jour d'ammoniac qui est de 5.7 GW qui équivaut à la production de 5 centrales nucléaires ou 2850 Ha de panneaux photovoltaics sans prendre en compte le pompage de l'eau. Il est donc inconcevable de remplacer le méthane pour produire de l'hydrogène par electrolyse de l'eau. Mais, nous pouvons contribuer à diminiuer la consommation de méthane et donc d'émettre moins de $\ce CO2$ en utilisant de l'electricité produit de manière propre (sans émettre de $\ce CO2$) par des énergies renouvelables telles que l'énergie éolien, solaire (à l'aide de panneaux photovoltaics),...  .

Un autre aspect qui nous pousserait à envisager ce mode de production est la production locale où les coûts de transports en seraient donc diminués. 

En termes de matières premières, le méthane est une source non-renouvelable qui s'épuisera d'ici une soixantaine d'années alors que l'electrolyse ne présente pas de problèmes d'approvisionnement (pour l'instant). L'electrolyse thermique, chaleur récupérée,...




\end{document}