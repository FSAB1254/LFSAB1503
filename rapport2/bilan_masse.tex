\documentclass[a4paper,12pt, oneside]{article}

\usepackage[in]{fullpage}

\usepackage[utf8]{inputenc}
\usepackage[T1]{fontenc}
\usepackage[french]{babel}

\usepackage{amsmath}
\usepackage{amssymb}
\usepackage{mathtools}
\usepackage[inline]{enumitem}
\usepackage[squaren,Gray]{SIunits}
\usepackage{sistyle}
\usepackage[autolanguage]{numprint}
\usepackage{xfrac}
\usepackage{bm}
\usepackage{color}
\usepackage[version=3]{mhchem}
\usepackage{multirow}
\usepackage{tabulary}

\newcommand{\diff}[1]{\mathrm{d}#1}

\let\oldvec\vec
\renewcommand{\vec}[1]{\oldvec{\bm{#1}}}
\newcommand{\uvec}[1]{\hat{\bm{#1}}}

\newcommand{\TODO}[1]{\colorbox{red}{\textbf{\textsc{TODO: #1}}}}

\newcommand{\e}[1]{\ensuremath{\cdot 10^{#1}}}

\makeatletter
\newcommand\reaction@[1]{\begin{equation}\ce{#1}\end{equation}}
\newcommand\reaction@nonumber[1]%
{\begin{equation*}\ce{#1}\end{equation*}}
\newcommand\reaction{\@ifstar{\reaction@nonumber}{\reaction@}}
\makeatother


%opening
\title{}
\author{Groupe 1254}
\date{01-10-2014}

\begin{document}

\section*{Bilan de masse}

Nous cherchons à produire une masse $m$ de \ce{NH3}. La masse molaire de la molécule étant de \unit{0.017}{\kilogram\per\mole}, cela est équivalent à un nombre de moles $n = m/(\unit{0.017}{\kilogram\per\mole})$ de \ce{NH3}. Pour cela, nous utilisons la réaction suivante :

\reaction*{1/2N2 + 3/2H2 -> NH3}

Nous en déduisons qu'il nous faut les quantités suivantes de \ce{N2} et de \ce{H2} :

\[
  n_\ce{H2} = \frac{3}{2} n 
  \qquad\text{et}\qquad
  n_\ce{N2} = \frac{1}{2} n
  \text.
\]

L'azote utilisé dans le réacteur provient directement de l'air, tandis que l'hydrogène est à produire à l'aide des réactions suivantes :
\begingroup
\addtolength{\jot}{.5em}
\begin{align}
  \cee{
     CH4 +  H2O &->  CO  + 3H2 \\
     CO  +  H2O &->  CO2 +  H2 \\
    2CH4 +  O2  &-> 2CO + 4H2 \\
    2CO  + 2H2O &-> 2CO2 + 2H2
  }
  \text.
\end{align}
\endgroup

Il est important de noter que ces réactions se produisent de façon séparées, l'une après l'autre. Une fois la quatrième réaction effectuée, les produits inutiles (\ce{CO2} et \ce{H2O}) sont rejetés, tandis que l'hydrogène et l'azote sont envoyés dans le réacteur pour la production d'ammoniac.

Nous allons maintenant déterminer la quantité de chaque réactif nécessaires afin d'obtenir dans le réacteur une composition st\oe{}chiométrique de diazote et de dihydrogène, telle que décrite par la réaction de production de l'ammoniac. Commençons par l'azote : celui-ci provient de l'air, dont la composition est \unit{78.08}{\%} de \ce{N2} et \unit{20.95}{\%} de \ce{O2}. Il est donc nécessaire d'injecter
\[
  n_\text{air} = \frac{n_\ce{N2}}{0.7808} = \frac{\frac{1}{2} n}{0.7808} = 0.6404 n
\]
moles d'air pour produire une quantité $n$ d'ammoniac.

Passons à l'hydrogène. Nous avons plusieurs réactions responsables de sa production et il est important de saisir les différences entre celles-ci. Les réactions (1-2) nécessitent un apport de méthane et d'eau que l'on peut aisément contrôler, tandis que les réactions (3-4) nécessitent notamment un apport d'oxygène qui, lui, est déterminé à l'avance par la composition de l'air et la quantité d'azote injectée dans le système. Nous allons donc déterminer quelle quantité d'hydrogène est produite par les réactions (3-4) à partir de l'oxygène disponible, puis nous compléterons le manque à l'aide des réactions (1-2).

Commençons par calculer la quantité de \ce{O2} disponible :
\[
  n_\ce{O2} = 0.2095 n_\text{air} = 0.2095 \times (0.6404 n) = 0.1342 n
  \text.
\]
Les équations (3-4) nous indiquent que chaque mole de \ce{O2} utilise \unit{2}{\mole} de \ce{CH4} et \unit{2}{\mole} de \ce{H2O} pour produire en tout \unit{6}{\mole} de \ce{H2} :
\begin{align*}
  n_{\ce{CH4}\text{ util. (3-4)}} &= 2 n_\ce{O2} = 0.2684 n \\
  n_{\ce{H2O}\text{ util. (3-4)}} &= 2 n_\ce{O2} = 0.2684 n \\
  n_{\ce{H2}\text{ prod. (3-4)}} &= 6 n_\ce{O2} = 0.8052 n
  \text.
\end{align*}

Il reste donc
\[
n_{\ce{H2}\text{ prod. (1-2)}} = n_\ce{H2} - n_{\ce{H2}\text{ prod. (1-2)}} = 0.6948 n
\]
de \ce{H2} à produire afin d'obtenir une composition st\oe{}chiométrique. Pour cela, nous allons utiliser les réactions (1-2).
Dans celles-ci, \unit{1}{\mole} de \ce{CH4} et \unit{2}{\mole} de \ce{H2O} sont consommées pour la production de \unit{4}{\mole} de \ce{H2}. Nous pouvons aisément calculer la quantité de réactifs nécessaires pour obtenir l'hydrogène manquant :
\begin{align*}
  n_{\ce{CH4}\text{ util. (1-2)}} &= \frac{1}{4} n_{\ce{H2}\text{ prod. (1-2)}} = 0.1737 n \\
  n_{\ce{H2O}\text{ util. (1-2)}} &= \frac{1}{2} n_{\ce{H2}\text{ prod. (1-2)}} = 0.3474 n
  \text.
\end{align*}

Nous pouvons maintenant résumer le bilan de masse :
\begin{center}
  \begin{tabular}{lccccc}
    & Air\footnotemark & \ce{CH4} & \ce{H2O} & \ce{H2} & \ce{NH3} \\
    \hline
    Réactions 1-2 &
     & $-0.1737 n$ & $-0.3474 n$ & $+0.6948 n$ & \\
    Réactions 3-4 &
    \multirow{2}{*}{$-0.6404 n$} & $-0.2684 n$ & $-0.2684 n$ & $+0.8052 n$ & \\
    Réac. finale &
     & & & $-1.500 n$ & $+1.000 n$ \\
    \hline
    Total &
    $-0.6404 n$ & $-0.4421 n$ & $-0.6158 n$ & & $+1.000 n$ \\
    Total (masse) &
    $-1.091 m$ & $-0.4161 m$ & $-0.6520 m$ & & $+1.000 m$
  \end{tabular}
\end{center}
\footnotetext{Composition : \unit{78.08}{\%} \ce{N2} et \unit{20.95}{\%} \ce{O2}, plus d'autres gaz en très faible proportions.}

\end{document}
