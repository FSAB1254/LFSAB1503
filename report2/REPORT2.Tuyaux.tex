\documentclass{article}
\usepackage[in]{fullpage}

\usepackage[utf8]{inputenc}
\usepackage[T1]{fontenc}
\usepackage[french]{babel}


%\usepackage{chemformula}
\usepackage{amsmath}
\usepackage{amssymb}
%\usepackage{mathtools}
\usepackage[inline]{enumitem}
\usepackage[squaren,Gray]{SIunits}
\usepackage{sistyle}
\usepackage[autolanguage]{numprint}
%\usepackage{xfrac}
\usepackage{bm}
\usepackage{color}
\usepackage[version=3]{mhchem}
\usepackage{multirow}
\usepackage{tabulary}
\usepackage{varwidth}

\newcommand{\diff}[1]{\mathrm{d}#1}

\let\oldvec\vec
\renewcommand{\vec}[1]{\oldvec{\bm{#1}}}
\newcommand{\uvec}[1]{\hat{\bm{#1}}}

\newcommand{\TODO}[1]{\colorbox{red}{\textbf{\textsc{TODO: #1}}}}

\newcommand{\e}[1]{\ensuremath{\cdot 10^{#1}}}

\makeatletter
\newcommand\reaction@[1]{\begin{equation}\ce{#1}\end{equation}}
\newcommand\reaction@nonumber[1]
{\begin{equation*}\ce{#1}\end{equation*}}
\newcommand\reaction{\@ifstar{\reaction@nonumber}{\reaction@}}
\makeatother

\begin{document}
\section*{Tuyaux}

Grâce au programme Matlab que nous avons écrit, nous sommes arrivés à déterminer le nombre de moles journalières de \ce{CH4} nécessaires pour pouvoir produire 1500 tonnes d'ammoniac par jour. A partir de cela, nous sommes parvenus à exprimer le débit de masse du \ce{CH4} qui est: 
	$$\dot{m}=\frac{n_{\ce{CH4}}\cdot M_{\ce{CH4}}}{24\cdot60\cdot60}$$
	$$\dot{m}=\frac{\unit{3.9027\cdot10^7}{mole}\cdot\unit{16\cdot10^{-3}}{\kilogram\per\mole}}{24\cdot60\cdot60}=\unit{7.24}{\kilogram\per\second}$$

On cherche dans un premier temps le flux d'écoulement de \ce{CH4}, $\dot{V}_{tot}$. On sait que: 
	$$\dot{V}_{tot}=\frac{\dot{m}}{\rho}$$

Il nous faut donc $\dot{m}$ (connu) et $\rho$. On sait aussi que:
	$$\rho = \frac{1}{v}$$

Selon la loi des gaz parfaits:
	$$pv=R^{*}T$$
	$$v=\frac{R\cdot T}{M\cdot p}$$
	$$v=\frac{\unit{8.314}{\joule\per\kelvin\mole}\cdot\unit{1080}{\kelvin}}{\unit{16\cdot10^{-3}}{\kilogram\per\mole}\cdot\unit{31}{bar}}=\unit{0.18}{\cubic\meter\per\kilogram}$$

Et donc:
	$$\rho = \frac{1}{v} = \unit{5.52}{\kilogram\per\cubic\meter}$$
	$$\dot{V}_{tot} = \frac{\dot{m}}{\rho}=\unit{1.31}{\cubic\meter\per\second}$$

Pour trouver le nombre de tuyaux nécessaires, il nous faut diviser le $\dot{V}_{tot}$ par le volume d'écoulement d'un seul tuyau $\dot{V}_{1t} = A\cdot c$. Connaissant le rayon d'un tuyau ($r=\unit{0.05}{\meter}$), on a:
	$$x = \frac{\dot{V}_{tot}}{\dot{V}_{1t}}$$
	$$x = \frac{\unit{1.31}{\cubic\meter\per\second}}{\unit{\frac{\pi}{200}}{\cubic\meter\per\second}}=\unit{83.39}{tuyaux}$$
	
Il nous faudrait donc entre 83 et 84 tuyaux pour satisfaire le besoin en \ce{CH4} dans le réacteur du reformage primaire.

\end{document}