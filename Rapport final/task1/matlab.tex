\section{Usage du programme Matlab}

Le programme Matlab de la tâche 1 se trouve dans le sous-dossier \texttt{/outil-gr54/}. Il s'agit de la fonction \texttt{manager(m\_NH3, T\_reformer)}, qui elle-même utilise d'autres fonctions, également présentes dans le répertoire. Une explication concise du fonctionnement de la fonction peut être trouvée en utilisant \texttt{help manager} dans la ligne de commande Matlab.

Il est également possible d'avoir recours à une interface graphique simple à l'aide du programme \texttt{interface}. Celle-ci permet d'entrer une quantité d'ammoniac et une température, et d'afficher le résultat dans un tableau.

Enfin, le programme \texttt{ammonia($n_n2$, $n_h2$, p, T)} permet de calculer le rendement d'un cycle de production d'ammoniac à partir d'une certaine quantité de \ce{N2} et de \ce{H2} à une pression $p$ et une température $T$. De nouveau, plus d'informations sont disponibles à l'aide de la commande \texttt{help ammonia}.
