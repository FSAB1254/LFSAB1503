\section{Atelier de créativité}
\subsection{Qu'est-ce que la créativité?}

La question de la définition de la créativité est compliquée --- y répondre demande de la créativité ! Cependant, nous avons tenté de formuler une réponse concise et utile, qui s'applique bien au contexte du travail de l'ingénieur :
\begin{quote}
    La créativité, c'est produire des solutions originales et efficaces à un problème bien posé.
\end{quote}
On y remarque deux mots-clés important: il s'agit de \emph{produire} et de \emph{problème bien posé}.

\subparagraph{Problème bien posé} Il est très important que l'objectif soit exposé sous la forme d'une question claire, comprise par tous et qui motive la recherche de solutions originale. Souvent, la difficulté d'une recherche de solution est de ne pas savoir où diriger l'impulsion créative. Lorsque c'est le cas, cela peut indiquer que la question à laquelle on tente de répondre n'est pas posée correctement. Par exemple, le problème peut être trop vague, trop restreint, pas assez motivant ou trop avancé par rapport à l'état actuel du projet.

\subparagraph{Produire} Afin de générer des idées innovantes, il vaut mieux privilégier en premier lieu la quantité sur la qualité, c'est-à-dire de ne pas se ruer sur la première solution qui semble correcte. Il est généralement impossible d'appréhender tout le potentiel d'une idée lorsqu'elle est énoncée, et souvent c'est la combinaison de plusieurs idées qui amène la meilleure solution\footnotemark. Dès lors, il devient important de produire une grande quantité de suggestions, à partir de laquelle il sera possible de travailler lors d'une étape ultérieure.

\footnotetext{C'est-à-dire la plus originale et la plus efficace.}

Mettons l'accent sur l'importance, lorsque l'on cherche à produire un grand nombre de pistes à explorer, d'accepter absolument toutes les idées. Lorsque l'on fait un brainstorming, on ne s'attarde pas sur la faisabilité ou l'intérêt des suggestions: tout ce qu'on veut, c'est de la quantité.

\subsection{Divergence-convergence: le cycle du processus créatif}

Le processus créatif peut être modélisé sous la forme d'un aller-retour incessant entre deux fonctionnements opposés mais complémentaires : la divergence et la convergence.

\subparagraph{Divergence} On rejoint ici le mot-clé \og{}produire\fg{} détaillé ci-dessus. En phase de divergence, on suggère, on liste, on organise les idées. Les différents membres de l'équipe s'écoutent chacun à leur tour, notent les propositions, rebondissent sur celles de leurs collègues et produisent des nouvelles suggestions à partir des précédentes. De nouveau, il est important de se souvenir que la quantité prime et que toutes les idées, même absurdes ou inutiles, sont les bienvenues.
\subparagraph{Convergence} Cela consiste à prendre les ressources produites dans une phase de divergence puis, tout en s'efforçant d'en conserver l'originalité, de se reconcentrer sur l'objectif afin d'obtenir des solutions plus concrètes.\\[0.4em]

Ce qui fait l'efficacité d'un processus créatif bien mené, c'est la complémentarité de ces deux phases fort différentes. L'une sans l'autre, on resterait soit incapable de choisir parmi un large panel de solutions abstraites, soit coincé dans un nombre trop restreint de possibilités peu originales.

Il est à noter que, trop souvent, on a tendance à rester coincé en phase de convergence. La divergence, surtout en groupe, peut susciter un malaise : on hésite à partager certaines idées que l'on pense stupides ou inutiles, on a peur de sortir de sa zone de confort. Souvenons-nous que la créativité est nourrie par l'inconfort et par le lâcher prise. Dès lors, en phase de divergence, il est important que l'environnement de réflexion soit totalement ouvert aux suggestions même les plus saugrenues.

\subsection{Techniques de créativité en équipe}

Une manière de chercher en équipe des solutions originale à un problème est de décomposer le processus en trois étapes :

\begin{enumerate}
    \item identifier clairement la problématique et en faire un objet de motivation ;
    \item générer des idées par convergence-divergence ;
    \item décider de la solution à appliquer.
\end{enumerate}

\subsubsection{Identifier la problématique}

Il s'agit de poser correctement la question à laquelle on souhaite répondre. En fonction de l'idée que l'on a de l'objectif, cette étape peut être simple et directe, comme elle peut être longue et compliquée.

Plusieurs techniques valent la peine d'être employées pour accompagner la réflection :
\begin{itemize}
    \item réaliser un mind-map autour du projet ou d'un certain aspect du projet ;
    \item effectuer des travaux de recherche ;
    \item travailler par divergence-convergence, établir une liste de questions, rebondir dessus puis identifier celle ou celles qui qualifient au mieux la demande.
\end{itemize}

\subsubsection{Générer des idées}

À ce stade, la problématique est correctement posée et on commence un brainstorming pour générer des idées. Il s'agit d'alterner entre phases de divergence, pour ouvrir la discussion, et de convergence, afin de la reserrer autour du problème et des enjeux réalistes.

Il peut être utile de travailler avec des post-it: on écrit dessus une description de chaque idée; ensuite, celles-ci peuvent être disposées, déplacées, regroupées, etc. Le processus créatif gagne toujours à obtenir une certaine dimension spatiale.

\subsubsection{Décider de la solution}

Lorsqu'une liste étoffée d'idées a été établie, la dernière étape consiste à choisir sur laquelle le groupe concentrera ses efforts. Cela revient généralement à sélectionner la solution la plus efficace, motivante et originale.

Peu importe la méthode de décision --- vote, choix du chef d'équipe, etc ---, il est important que tous les membres du groupe soient d'accord, à la fin du processus, avec la solution choisie et avec les raisons de ce choix.


\subsubsection{Proposer l'idée aux clients}

Une fois la meilleure solution sélectionnée, il reste à \og{}vendre\fg{} celle-ci aux clients, afin de les convaincre qu'elle représente bien le choix le plus avantageux. Pour cela, il est nécessaire de produire une présentation qui identifie et synthétise les quatres concepts suivants:

\begin{enumerate}
    \item les besoins des clients ;
    \item la réponse apportée à ces besoins ;
    \item la particularité, l'originalité de la solution proposée ;
    \item une explication résumée de cette solution.
\end{enumerate}

Les étapes 1-3 servent à convaincre le client de la qualité de la solution, tandis que l'étape 4 permet de résumer et de mémoriser celle-ci.

En outre, il est important de soutenir le propos, tout au long de la présentation, avec des arguments : expliquer l'origine des idées, le chemin parcouru pour en arriver à la solution choisie, les recherches effectuées, etc.












