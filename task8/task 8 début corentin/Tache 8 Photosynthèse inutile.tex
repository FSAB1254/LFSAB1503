\documentclass[10pt,a4paper]{article}
\usepackage[utf8]{inputenc}
\usepackage[francais]{babel}
\usepackage[T1]{fontenc}
\usepackage{amsmath}
\usepackage{amsfonts}
\usepackage{amssymb}
\usepackage{graphicx}
\usepackage[squaren,Gray]{SIunits} % Physical units rendering
\usepackage{sistyle}
\usepackage[autolanguage]{numprint}
%\usepackage{xfrac}
\usepackage{bm}
\usepackage{color} % Colors in text
\usepackage[version=3]{mhchem} % Chemical reactions
\title{Tâche 8 : Photosynthèse}
\author{Goyens Virgile \and Joachim Corentin}
\begin{document}
\maketitle
%Tâche 8 : Amélioration du procédé
%Sur base des résultats et informations obtenus dans les autres tâches, le groupe analysera de façon créative les possibilités de diminuer globalement la consommation d’énergie ou de réduire l’impact environnemental de la production. Il proposera et détaillera une amélioration possible du procédé. Le choix de l’amélioration proposée sera argumenté et son impact sera quantifié.

D'un point de vue environnemental, il est essentiel de diminuer le plus possible notre consommation d'énergie ainsi que de réduire l'impact sur l'environnement de la production. Nous allons donc étudier une des méthodes envisagées pour traiter le \ce{CO2} émis dû à la production d'ammoniac, à savoir la photosynthèse.

\section{Photosynthèse:}
Pour rappel, la photosynthèse est le processus permettant aux plantes et à certaines bactéries de synthétiser du glucose, en partant de \ce{CO2}, lumière et sels minéraux.

Equation de la photosynthèse:
\ce{6CO2 + 6H20 \text{+ lumière} <=> C6H1206 + 6O2}

\begin{itemize}
\item Avantage(s): 
	\begin{itemize}
	\item facile à réaliser
	\end{itemize}
\item Désavantage(s): 
	\begin{itemize}
	\item Besoin d'énergie (lumière en continu)
	\item Besoin d'espace
	\end{itemize}
\end{itemize}

\section{Planter des arbres près du site de production?}

Il faut savoir que plus un abre est vieux moins il va nous être utile dans le traitement du \ce{CO2}. En effet, un vieil arbre soutire presque autant de \ce{CO2} qu'il n'en émet.

Il est difficile de donner une valeur chiffrée précise quant à la quantité de \ce{CO2} qu'un abre peut absorber. En effet, ceci dépend d'une multitude de facteurs, à savoir: son emplacement, son espèce, l'ensoleillement, sa taille, son age ... Cependant, selon les chiffres, un arbre = 10kg de \ce{CO2} / année en moyenne.
Or, pour la production quotidienne de 1500t de \ce{NH3} nous rejettons 1925t de \ce{C02}.
Sur base de ces données calculons le nombre d'abres ainsi que la superficie associée à une telle production pour voir si le projet est réalisable.

\ce{C02} : $$1.925.000 * 365 = 702.625.000kg/an $$ 
Ce qui représente approximativement 70.262.500 arbres soit plus ou moins 100.000 hectares.



\section{Sources :}

\begin{itemize}
\item http://www.apcas.qc.ca/wp-content/uploads/2011/12/2011fevbertin.pdf
\item http://www.zeroco2.com/blog/2012/07/05/quel-est-le-potentiel-dabsorbtion-de-ges-dun-arbre/
\end{itemize}


\end{document}