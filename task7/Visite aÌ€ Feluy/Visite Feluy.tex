\documentclass[10pt,a4paper]{article}
\usepackage[utf8]{inputenc}
\usepackage[francais]{babel}
\usepackage[T1]{fontenc}
\usepackage{amsmath}
\usepackage{amsfonts}
\usepackage{amssymb}
\usepackage{graphicx}
\begin{document}
\section{Visite de Total à Feluy}

Nous avons visité l'entreprise Total située à Feluy, c'est une entreprise pétrolière qui s'occupe de produire et fournir de l'énergie. Notre visite s'est déroulée en différentes étapes, tout d'abord une petite présentation sur l'entreprise elle-même ainsi que sa place dans le marché mondial. Ensuite, nous avons pu visiter différentes parties de l'entreprise ainsi que l'unité pilote qui leur permet de réaliser des tests à plus petite échelle afin de ne pas bloquer l'entiereté de la production en cas de tests défayants.

\subsection{Le rôle des catalyseurs}

Les catalyseurs jouent un rôle essentiel dans une grande partie des réactions chimiques. En effet, ceux-ci permettent d'augmenter la vitesse de la réaction ainsi que son rendement, ce qui est primordial dans la chaîne production d'une entreprise. Le catalyseur va affaiblir la molécule qui réagit, ce qui va permettre à la réaction de se dérouler plus rapidement.
On place souvent ces catalyseurs sur un support poreux ce qui a pour effet de sélectionner uniquement certaines molécules. Les catalyseurs sont également très utiles pour diriger une réaction lorsque qu'un réactif peut réagir de diverses manières, cela permet de diminuer les pertes vers d'autres réactions. Cependant, il faut savoir qu'il existe des réactions où le catalyseur est indispensable, sans lui rien ne se passe.

\subsection{Sécurité}

Dans une entreprise de telle ampleur, il est indispensable d'avoir des mesures de sécurité à respecter pour la santé de la population ainsi que du point de vue environnemental. Il existe donc une multitude de précotions à prendre. Sur le site de Feluy, nous avons par exemple remarqué ces tours qui montent à une centaine de mètres de hauteur, elles permettent en cas de problèmes, de brûler les composants et de les faire sortir vers l'exterieur. La hauteur se justifie par la quantité de chaleur qui est dégagée lors de ce processus.
\end{document}