\documentclass[a4paper,12pt, oneside]{article}

\usepackage[in]{fullpage}

\usepackage[utf8]{inputenc}
\usepackage[T1]{fontenc}
\usepackage[english]{babel}

\usepackage{amsmath}
\usepackage{amssymb}
\usepackage{mathtools}
\usepackage[inline]{enumitem}
\usepackage[squaren,Gray]{SIunits}
\usepackage{xfrac}
\usepackage{bm}
\usepackage{color}

\newcommand{\diff}[1]{\mathrm{d}#1}

\let\oldvec\vec
\renewcommand{\vec}[1]{\oldvec{\bm{#1}}}
\newcommand{\uvec}[1]{\hat{\bm{#1}}}

\newcommand{\TODO}[1]{\colorbox{red}{\textbf{\textsc{TODO: #1}}}}


\begin{document}

\part*{Activité de terrain: visite de Yara à Tertre}
\section{Sécurité}
	\subsection{Risques encourus}
		Premièrement, on retrouve dans la production d'ammoniac deux réactifs sous forme de gaz: l'azote, \ce{N2}, et l'hydrogène, \ce{H2}. Ces gaz ont la particularité d'être à la fois incolores et inodores. Le premier présente un danger de par le fait qu'il va "prendre la place" de l'oxygène présent dans l'air, provoquant ainsi un risque d'asphyxie (un taux d'oxygène inférieur à 19\% est létal). 
		
		L'hydrogène, quant à lui, est d'une part explosif (une fuite d'hydrogène augmentant donc grandement le risque d'incendies!) et d'autre part fortement corrosif envers les couches de fer composant les parois des réacteurs et reformeurs. Ainsi, ce composé crée des lèvres et des fissures dans lesdites parois, provoquant l'échappement des gaz présents dans le procédé chimique.
		
		En second lieu, l'ammoniac qui est lui aussi sous forme gazeuse, implique bien des risques. Bien qu'il soit facilement repérable grâce à son odeur particulière et prononcée (odeur repérable à partir de 5 ppm), il représente de gros risques pour la santé. En effet, une concentration d'\ce{NH3} comprise entre 400 et 700 ppm provoquera chez un inhalateur des irritations de la gorge variant en intensité tandis qu'une concentration de 10 000 ppm provoquera de graves conséquences médicales.
		
		Par ailleurs, l'ammoniac peut devenir explosif sous certaines conditions particulières de température et de pression. Il attaque également le zinc et le cuivre, nécessitant un équipement approprié lors de sa manipulation. Finalement, le \ce{NH3} présent sous l'état liquide s'évapore très rapidement lorsqu'il entre en contact avec de l'eau: il faudra donc pratiquer le \og spinning\fg \ si nous sommes amenés à nettoyer une \og flaque\fg \ d'ammoniac.
		
		Enfin, divers risques viennent se rajouter à ceux créés par la manipulation de produits chimiques. On compte notamment le travail en hauteur ainsi que la présence de réacteurs à haute pression.
		
	\subsection{Mesures de sécurité}
		Vu le nombre de risques présents sur le site de Yara, il est normal que plusieurs mesures de sécurité soient prises pour éviter autant que faire se peut les accidents mortels. 
		
		On distingue avant tout deux types de signaux sonores d'une durée de deux minutes. D'abord, un signal sonore continu synonyme d'un risque modéré. Les mesures à prendre dans un tel cas sont de rentrer dans un bâtiment et de se conformer aux consignes  données. 
		
		Le deuxième type d'alarme est un son ondulé. Il est synonyme de problème majeur et nécessite de se rassembler immédiatement aux lieux prévus à cet effet. Ces deux sirènes sont testées chaque premier jeudi du mois pour s'assurer de leur bon fonctionnement.
		
		Chaque personne pénétrant le site de production se voit aussi attribuer un masque de fuite sous vide à utiliser en cas de fuite de gaz toxique et/ou d'odeur anormale. Ce masque est adapté à tous les fluides néfastes présents sur le site. En outre, chacun doit aussi être muni d'un casque de chantier, de lunettes de protection, de chaussures solides et éventuellement, d'une forme d'isolation sonore.
		
		Il y a aussi plusieurs interdictions à respecter lors de la durée de visite du site, permettant une sécurité optimale:
			\begin{itemize}
			\item interdiction de consommer de la drogue et/ou de l'alcool
			\item interdiction de fumer
			\item interdiction d'utiliser un téléphone portable
			\item interdiction de prendre des photos
			\item respecter les panneaux de signalisation
			\item respecter les emplacements de parking
			\end{itemize}
		
		Finalement, Yara possède quatre règles d'or qui ne doivent en aucun cas être enfreintes par ses employés. Elles sont:
			\begin{enumerate}
			\item Utilisation du harnais lors du travail en hauteur
			\item Port des équipements de protection lors de manipulation de substances dangereuses
			\item Stricte régulation de l'état des systèmes de contrôle - sécurité.
			\item Précaution d'emploi lors de travail avec les sources d'énergie
			\end{enumerate}
		
		Tous ces aspects sécuritaires sont gérés via deux centres de contrôles où est minutieusement répertorié et analysé l'état des différentes unités du centre de production.
		
		En outre, pour des raisons légales ainsi que de maintenance, Yara effectue un arrêt complet de l'unité de production une fois tous les quatre ans.
		
\section{Environnement}
	Au vu de l'intérêt grandissant pour une industrie propre de tout déchet nocif pour l'environnement, Yara  s'efforce de prendre certaines mesures pour minimiser son empreinte écologique. 
	
	Malheureusement, il subsiste quelques sources de pollution chez Yara: le rejet de \ce{CO2}, de solution de décarbonatation et de composés aminés, les fumées issues du réformeur primaire et la consommation d'huile par les machines tournantes.
	
	Pour palier à une de ces sources de pollution, Yara a donc décidé de recycler le \ce{CO} produit grâce à plusieurs étapes:
		\begin{enumerate}
		\item Conversion du \ce{CO} en \ce{CO2} (car plus facile à éliminer)
		\item Purification du \ce{CO2} à l'aide de strippers
		\item Rejet du \ce{CO2} impur
		\item Liquéfaction du \ce{CO2} pur
		\item revente du \ce{CO2} pur liquéfié
		\end{enumerate}
		
\section{Aspect technique}
	Il est évident que dans le cadre de notre projet, nous avons effectué moult simplifications dans le procédé de production de \ce{NH3}. Dans un souci de concision, nous ne mentionnerons qu'un élément que nous avons omis pour notre projet: la boucle de circulation.
	
	La boucle de circulation est située dans le réacteur de synthèse de l'ammoniac. Elle réinjecte les réactifs non consommés dans le réacteur. Cela permet une optimisation du processus et ainsi un cout de production minimal et un rendement maximal. Toutefois, il faut aussi insérer une purge dans cette boucle de circulation pour éviter une accumulation excessive de gaz indésirables comme l'argon et l'hélium.
		
	Mentionnons aussi l'utilisation d'un catalyseur lors de la phase finale de la synthèse d'ammoniac. Ce catalyseur, un alliage composé de magnétite et de fer-111, est présent dans le réacteur sous forme de sphères. Le principe de ce catalyseur est de servir de site de fixation aux molécules pour leur permettre de réagir plus facilement. Il diminue donc l'énergie d'activation de la réaction de formation du \ce{NH3} mais ne change pas le chemin réactionnel!
	
	Nous conclurons ce rapport par quelques données techniques:
	\begin{itemize}
	\item le reformeur primaire de Yara fonctionne à une température de 850\celsius \ et à une pression de 33 bar.
	\item actuellement, Yara fourni environ 40\giga\joule \ d'énergie afin de produire une tonne de \ce{NH3}.
	\item Yara, c'est 1150\tonne \ de \ce{NH3} par jour. 
	\end{itemize}
		
		

\end{document}