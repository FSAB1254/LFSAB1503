\documentclass{beamer}
\usepackage[utf8]{inputenc}
\usepackage[francais]{babel}
\usepackage[T1]{fontenc}
\usepackage{graphicx}
\usepackage{amsmath}
\usepackage{amsfonts}
\usepackage{amssymb}
\usepackage{graphicx}
\usepackage{tikz}
\usetikzlibrary{arrows}
\usepackage[squaren,Gray]{SIunits} % Physical units rendering
\usepackage{sistyle}
\usepackage[autolanguage]{numprint}
%\usepackage{xfrac}
\usepackage{bm}
\usepackage{color} % Colors in text
\usepackage[version=3]{mhchem} % Chemical reactions
\usetheme{CambridgeUS}
\setbeamercolor{title}{bg=red!65!black,fg=white}

\setbeamertemplate{sidebar right}
{
  \vfill%
  \llap{\insertlogo\hskip0.1cm}%
  \vskip2pt%
  \llap{\href{http://tex.stackexchange.com/}{A link to tex.sx}\hskip0.2cm}% NEW
  \vskip3pt% NEW
  \llap{\usebeamertemplate***{navigation symbols}\hskip0.1cm}%
  \vskip2pt%
}

\begin{document}
\begin{frame}{Quid du biogaz comme alternative au méthane pur?}
\part*{Avantages}
\begin{itemize}
\item Au total, pas de rejet de \ce{CO2}
\item Diminution de la quantité de \ce{CH4} présent dans l'atmosphère
\item Moins de pollution liée à la recherche de méthane pur
\item Problèmes de transport évités si couplage d'usines
\end{itemize}
\end{frame}

\begin{frame}{Faisabilité}
\begin{itemize}
\item Pour 1500\tonne \ de \ce{NH3} quotidiennes, il nous faut 1 509 188\cubic\meter \ de biogaz
\item En région wallonne, on pourrait récolter assez de déchets afin de produire 86.6 millons de \cubic\meter \ de biogaz
\item Le biogaz ne pourra donc pas totalement remplacer le méthane pur dans un premier temps
\end{itemize}

\end{frame}
\end{document}