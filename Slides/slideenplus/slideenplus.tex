\documentclass{beamer}
\usepackage[utf8]{inputenc}
\usepackage[francais]{babel}
\usepackage[T1]{fontenc}
\usepackage{amsmath}
\usepackage{amsfonts}
\usepackage{amssymb}
\usepackage{graphicx}
\usepackage{tikz}
\usetikzlibrary{arrows}
\usepackage[squaren,Gray]{SIunits} % Physical units rendering
\usepackage{sistyle}
\usepackage[autolanguage]{numprint}
%\usepackage{xfrac}
\usepackage{bm}
\usepackage{color} % Colors in text
\usepackage[version=3]{mhchem} % Chemical reactions
\usetheme{CambridgeUS}
\setbeamercolor{title}{bg=red!65!black,fg=white}

\setbeamertemplate{sidebar right}
{
  \vfill%
  \llap{\insertlogo\hskip0.1cm}%
  \vskip2pt%
  \llap{\href{http://tex.stackexchange.com/}{A link to tex.sx}\hskip0.2cm}% NEW
  \vskip3pt% NEW
  \llap{\usebeamertemplate***{navigation symbols}\hskip0.1cm}%
  \vskip2pt%
}

\begin{document}

\title{Synthèse de l'ammoniac}
\author{Groupe 1254}
\institute[UCL]{Ecole polytechnique de Louvain-la-neuve}
\date{}
\maketitle
%==========================================================

\begin{frame}{Analyse du progrès du groupe}

		\begin{center}
		Organisation du groupe:
			\begin{itemize}
			\item Utilisation de Github.
			\item Planification par écrit des tâches.
			\item Réservation de Locaux en BST.
			\end{itemize}
		\end{center}

\end{frame}

\begin{frame}[allowframebreaks]{Le biogaz en Wallonie}
		\begin{center}
\scalebox{0.8}{
\begin{tabular}{|c|c|c|}
\hline 
 & Gisement ($\unit{10^6}{t}$) & Productivité ($\unit{}{m^3_{\ce{CH4}}/t)}$\\ 
\hline 
Effluents agricoles & 18.2 & 31.5  \\ 
\hline 
Résidus agro-industriels & 1.15 & 60 \\ 
\hline 
Résidus organiques ménagers + déchets verts & 1 & 65\\ 
\hline 
Boues de STEP& 0.07 & 230 \\ 
\hline 
Total & 20.42 & \\ 
\hline 
\end{tabular} 
}
\end{center}


A partir de ces données, nous pouvons faire un estimation de la production de biométhane en Wallonie:

$$18.2\cdot \num{e6} \cdot 31.5 + 1.15\cdot \num{e6} \cdot 60 + 1\cdot \num{e6} \cdot 65 + 0.07\cdot \num{e6} \cdot 230 = \unit{729.4\cdot \num{e6}}{m^3}$$ en sachant que le masse volumique du \ce{CH4} est de \unit{0.6790}{kg/m^3},
on obtient que la combinaison de ces 4 ressources, nous engendre une production de $\unit{485.33 \cdot \num{e3}}{t/an}$ de \ce{CH4}.

Comme nous avons besoin de \unit{708.76}{t/day} de \ce{CH4}, il nous faut \unit{258697.5}{T/ans} de \ce{CH4}. Ce qui équivaut à \unit{53.3}{\%} de la production de biométhane en Wallonie.

\end{frame}


\end{document}