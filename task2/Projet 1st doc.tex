\documentclass[10pt,a4paper]{article}
\usepackage[utf8]{inputenc}
\usepackage[francais]{babel}
\usepackage[T1]{fontenc}
\usepackage{amsmath}
\usepackage{amsfonts}
\usepackage{amssymb}
\usepackage{graphicx}
\usepackage[squaren,Gray]{SIunits} % Physical units rendering
\usepackage{sistyle}
\usepackage[autolanguage]{numprint}
%\usepackage{xfrac}
\usepackage{bm}
\usepackage{color} % Colors in text
\usepackage[version=3]{mhchem} % Chemical reactions
\begin{document}

\section{Synthèse \ce{NH3} et séparation}

Nous allons maintenant étudier un peu plus en détail la dernière étape du procédé qui comprend le réacteur de synthèse d’ammoniac et une étape de séparation des produits.
La synthèse de \ce{NH3} s'écrit :
$$ \ce{N_2 + 3H_2 \leftrightarrow 2NH_3}$$

\subsection{Manière qualitative}
Voyons maintenant de quelle manière évolue la réaction en faisant varier d'une part la température, et d'autre part la pression. Pour cela, nous nous basons sur le fait qu'une réaction chimique aura toujours tendance à vouloir aller à l'encontre des modifications qui lui sont apportées, à savoir le principe énoncé par le Chatelier. Dans notre cas, nous devons favoriser la réaction allant dans le sens de production de \ce{NH3}.

Tout d'abord, regardons ce qu'il en est au niveau de la pression. Nous devons faire augmenter celle-ci, car de cette manière le système réagira en favorisant le sens de réaction produisant le plus petit nombre de moles de gaz, à savoir \ce{NH3} dans le but de rétablir la pression. 

Ensuite, nous allons étudier le comportement de notre réaction lorsqu'on modifie la température. Pour cela, nous devons déterminer si la réaction est soit endothermique soit exothermmique. Nous calculons donc le $\Delta H$  de la réaction. $$\Delta H = \unit{-92,4}{kJ.mol}$$
Le signe de $\Delta H$ étant négatif, notre réaction est bien exothermique. De ce fait, nous devons diminuer la température afin que l'on favorise la réaction à dégager de la chaleur, et donc à se déplacer vers la droite.

Nous en venons donc à la conclusion qu'il nous faudrait ne pression théoriquement infinie et une température la plus basse possible. Cependant, nous sommes restreint économiquement et techniquement, nous devons donc trouver un juste milieu, une conversion totale n'est donc pas envisageable. 

\subsection{Manière quantitative}

Variation de la température $\frac{\delta G}{\delta t}$ :
$$\Delta G = \Delta H - T \Delta S $$
$$ avec: \Delta H = \unit{-92,4}{kJ.mol} $$ \
$$ \Delta S = \unit{-198,7}{kJ.mol} $$

% + SUITE

\end{document}